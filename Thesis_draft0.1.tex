\documentclass[11pt, a4paper, twoside]{report}
\usepackage[utf8]{inputenc}
\usepackage{graphicx}
\usepackage{amsmath}
\usepackage{titlesec}
\titleformat{\chapter}{\lmodern\it\huge\bfseries\flushright}{}{1cm}{}
\titleformat{\section}{\lmodern\Large\bfseries}{}{0.5cm}{}
\titleformat{\subsection}{\lmodern\normalsize\bfseries}{}{0cm}{}
\titleformat{\subsubsection}{\lmodern\small\bfseries}{}{0cm}{}
\graphicspath{{/home/chris/Dropbox/HOTOSM/figures/}}
\usepackage{fancyhdr}
\pagestyle{fancy}
\pagenumbering{roman}

\begin{document}

\begin{titlepage}
   \begin{center}
       \vspace*{1cm}

       \huge
       \textbf{Thesis Title}

       \vspace{0.5cm}
       \Large
       Thesis Subtitle

       \vspace{1.5cm}

       \textbf{CHAN, Yan-Chak Christopher}

       \vspace{0.5cm}
       \textbf{Supervised by:} \\
       Prof. Dr. Hannes Taubenöck \thanks{Geo-Risks and Civil Society, Deutsches Zentrum für Luft- und Raumfahrt} \\
       Emran Alchikh Alnajar \thanks{Humanitarian OpenStreetMap}

       \vfill

       Masterarbeit submitted for the degree of\\
       Master der Naturwissenschaften\\
       in\\
       Applied Earth Observation and Geoanalysis of the Living Environment (EAGLE)

       \vspace{0.8cm}

       \includegraphics[scale = 0.25]{neuSIEGEL.png}

       \normalsize
       Philosophische Fakultät (Historische, Philologische, Kultur- und Geographische Wissenschaften)\\
       Julius-Maximilians-Universität Würzburg\\
   \end{center}
\end{titlepage}

\newpage

\section{Forewords and Acknowledgements}
\pagestyle{empty}

\newpage

\section{Declaration of Independent Work}
\pagestyle{empty}

\newpage

\section{Figure list}
\pagestyle{empty}

\newpage

\section{Abbreviations}
\pagestyle{empty}

\newpage

\begin{abstract}

Lorem ipsum dolor sit amet, consectetur adipiscing elit. Curabitur dignissim, quam maximus posuere cursus, magna justo rutrum erat, at mattis magna magna nec risus. Duis lacus lectus, condimentum a viverra eu, fermentum molestie lorem. Sed maximus, enim eu scelerisque dictum, sem erat mollis massa, in dictum ante libero a tortor. Cras nulla nisi, sollicitudin ac suscipit cursus, maximus non dui. Sed venenatis ligula id efficitur imperdiet. Vivamus ut magna eleifend, rutrum ante facilisis, pulvinar turpis. Maecenas at interdum lorem. Duis vel varius ligula. Sed magna erat, egestas vitae varius id, cursus vitae neque. Orci varius natoque penatibus et magnis dis parturient montes, nascetur ridiculus mus. Phasellus interdum lectus mi, a dapibus lorem tristique a. Phasellus molestie vestibulum metus a fringilla. Pellentesque at rhoncus nulla. Praesent posuere turpis nec leo fringilla egestas.\\\par

Pellentesque auctor vel dolor eu viverra. Ut faucibus nunc orci, eu aliquam justo hendrerit vel. Proin auctor sed nisl non posuere. Vivamus orci orci, commodo eget semper nec, tempus at arcu. Nam eget leo cursus velit aliquam varius. Curabitur nisi dui, rutrum vitae elit a, mollis volutpat mauris. Suspendisse potenti. Nam convallis magna iaculis posuere aliquam. Quisque tristique rutrum placerat. Quisque ultricies molestie lacinia. Maecenas at nisi in neque dictum consequat.\\\par

\end{abstract}

\newpage

\pagenumbering{arabic}

\tableofcontents

\newpage

\chapter{Introduction}\label{Intro}

The world’s population is more urbanised than ever before. As of 2018, approximately 4 billion (55\%) (UN DESA., 2018, Taubenöck et al., 2009) reside in urban areas, of which 60\% reside in slums often located at the fringes of the city (Venables A., 2018). Urbanisation growth is expect to increase by 2.5 billions between 2018 to 2050, most of which will be in Asia and Africa (UN DESA., 2018).When population growth outpace development, slums became the supplier of significant housing stocks. These informal settlements are dynamic and represent a good reflection of cultural practices, access to resources, financial limitations and other socio-economic conditions. This means the informal settlement differs significantly between urban and rural settlements of roof covers, densities, and are subjected to different levels of access to resources and the types of resources.\\\par

Refugee camps are often the common or only way for displaced people to receive shelters and assistance. They are often setup in place of proximity to displaced population, whether that be from natural disasters, human caused disasters, or other reasons. Throughout history, refugee sites have provided haven to the world's most vulnerable population (UN, 2018, Turner S., 2016, UNHCR, 2021). However as of 2020, out of the 26.4 million refugees, only around 1.4 million have access to third country solution between 2016 to 2021 (UNHCR, 2021). Additionally, although officially defined as temporary settlement, many refugee camps have had longer than expected life cycle, some of them have even became "Secondary Cities" and therefore suffers similar problems of poor urban governance and consequentially unattractive as investment (Cities Alliance \& AfDB., 2022).

\section{Study Area of Interest}\label{StudyAOI}
\subsection{Kalobeyei, Kakuma, Turkana, Kenya}\label{Kalobeyei}
\subsection{Dzaleka, Dowa, Malawi}\label{Dzaleka}
\subsection{Research Questions}\label{RQ}

\newpage

\chapter{Literature Review}\label{LitReview}
\section{Remote Sensing of Informal Settlements}\label{RSofInformalSettlement}
Informal settlement and slums mapping of developing countries require very high resolution (VHR) images which was unavailable until the turn of the century. The relatively new technology thus only began to gain traction within the last 2 decades. Particularly with the increase in the availability of VHR satellites. Increase in computational power had enabled novel techniques such as multi-layer machine learning, textural analysis, and novel geostatistical methods to emerge (Kuffer et al., 2016). Due to frequent repeat coverage of satellite’s orbit, they can be used to fill in between periodic census that are costly and time consuming to conduct. Census also does not do a good job in capturing larger scale units and spatial patterns, potentially overlooking others socioeconomic determinants such as cropping cycles and infrastructure access etc.Availability of VHR sensors and publications on slums and remote sensing (Kuffer et al., 2016). Remote sensing of settlements largely falls under 2 categories, rural or urban. Due to the different make-up of socioeconomic context and urban morphology, sensing rural and urban settlements require different parameters. Additionally, there’s no “one size fit all” way to generalise informaland formal settlement across the world, as physical geography, topography, cultural, and available resources often determine the distribution, development, and settlement clusters pattern

\section{Deep Learning in Remote Sensing}\label{DLinRS}

\subsection{Computer Vision and Convolutional Neural Networks}\label{CVandCNN}

The issue with any Deep Learning Project is the high amount of data required (Tan et al., 2018, )

\subsection{Computer Vision in Building Segmentation}\label{CVinBS}

\newpage

\chapter{Data and Methodologies}\label{DataandMethods}
\section{Data}\label{Data}

\subsection{Raster pre-processing}
\subsubsection{Normalisation}

\subsubsection{Cropping}

\subsection{Data Augmentation}\label{DataAug}

Data augmentation is perhaps one of the most crucial task in training a robust neural-network. It is an economical way of increasing generalisability without increasing model complexity, data augmentation achieve this through, firstly increasing the quantity of training and validation data, secondly encompassing a greater range of textural, geometrical, and colour variability throught the creation of augmented pseudo-data (Shorten \& Khoshgoftaar, 2019; Kinsley \& Kukiela, 2020; Howard \& Gugger, 2020; Zoph et al., 2019).\\\par

Data augmentation can generally be split into 3 categories: 1. Geometric/Affine distortion, 2. Colour distortion, and 4. Noise distortion. The application of which types of distortion to the \textit\{Train} and \textit\{Validation} dataset is highly dependent on the context of the semantic task. Therefore, care must be taken as to not introduce mislabelling \textit{(see Figure \ref{fig:MNIST5})} (Ng A., 2018).\\\par

\textbf{Augmentation categories:}

\begin{itemize}
  \item Geometric/Affine distortion
    \begin{itemize}
      \item e.g. Fliping, Stretching, Rotation...
    \end{itemize}
\end{itemize}
\begin{itemize}
  \item Colour distortion
    \begin{itemize}
      \item e.g. Colour Inversion, Solarise Colour, Greyscale...
    \end{itemize}
\end{itemize}
\begin{itemize}
  \item Noise distortion
    \begin{itemize}
      \item e.g. Blurring, Contrasting, Salt \& Pepper...
    \end{itemize}
\end{itemize}


\begin{figure}[h]
  \centering
  \includegraphics[scale = 0.5]{MNIST5.png}
  \caption{Perhaps geometric augmentation of horizontal flipping shall not be applied on the MNIST number of 5}
  \label{fig:MNIST5}
\end{figure}


\section{Architecture and hyperparameter selection}\label{Archi}

Model architecture and their associated hyperparameters selection is highly dependent on the computational resources and the task at hand (Ng A., 2018, Howard \& Gugger, 2020). As this study aims to output a pixel-based binary segmentation which delineates building and non-building, and given the computational resources constraint, tried and tested architectures with relatively low number of training parameters is ideal.

\subsection{The U-Net and U-Net variants}\label{Unet}

The U-Net architecture was first developed by Ronneberger et al. (2015) for the task of cell segmentation in biomedical electronmicroscope images. The architecture feature a symmetrical Encoder-Decoder structure \textit{(see figure \ref{U-Net})} and as with many other CNN, the architecture have transferred successfully well into the remote sensing domains (Höser \& Künzer, 2020, Höser et al., 2020) (e.g. Jean et al., 2016, Xu et al., 2019)

\subsection{Pre-trained weights on Deep Learning models}

Another major consideration of this study is to compare the performance of the architectures on various pre-trained weights. Due to the representation learning feature of CNN, studies have shown transfer learning on pre-trained weights even across different domain dataset could result in better performance, especially on projects with less data availability (LeCun et al., 2015, Zhu et al., 2017, Tan et al., 2018). Pre-trained weights for this study are ImageNet weights and the drone-based building segmentation competition (OCC) weights. Thus, one of the objective of this study is to find out whether transfer training performance on various pretrained weights would outperform training from scratch.

\begin{figure}[h]
  \centering
  \includegraphics[scale = 0.25]{U-Net.png}
  \caption{The Encoder-Decoder U-Net architecture (Ronneberger et al., 2015)}
  \label{fig:U-Net}
\end{figure}


\subsubsection{EfficientNet encoders}

\section{Accuracy Assessment}\label{AccAss}

Detail and scrutable accuracy assessments are fundamental towards any classification based analysis. This section will introduce and break down the various lower order and higher order class-based (thematic) accuracy assessment. By explaining the characteristics of each metrics, this will provide a much more granular nature of accuracy assessment in the findings of section \ref{findings}. In general, accuracy assessment in remote sensing can be divided into 2 categories: 1. Positional Accuracy \& 2. Thematic Accuracy. Of which, Positional Accuracy deals with the accuracy of the location while Thematic Accuracy deals with the labels or attributes accuracy (Congalton \& Green, 2019 \& Bolstad, 2019). The rest of this section will consider the lower order and higher order accuracy metrics, with lower order metrics being more granular while higher order metrics more triturated but generalised.\\\par

The metrics described in this section form part of the larger family of accuracy assessment metrics that can be constructed from the confusion matrix \textit{(see Figure \ref{fig:cmatrix})}

\begin{figure}[h]
  \centering
  \includegraphics[scale = 0.25]{cmatrix.png}
  \caption{The Confusion Matrix}
  \label{fig:cmatrix}
\end{figure}

\subsection{Precision, Recall, Sensitivity, and Specificity}\label{1storder}
\subsubsection{Precision, Recall, and Specificity}\label{PR&S}

\textbf{Precision} and \textbf{Recall}, aka. Positivie-Predictive-Value and Sensitivity/True-Positive-Rate Respectively. The two metrics are often used together, another common denomination especially in remote sensing literature are User's Accuracy and Producer's Accuracy (Congalton \& Green, 2019 \& Wegmann et al., 2016). To avoid further confusion in nomenclature, \textbf{Precision} and \textbf{Recall} will be used from hereon.\\\par
\\
\\
\textbf{Precision} is the measure of correctly predicted Positive class (True Positive) against all positive prediction assigned to that class (True Positive + False Positive) i.e. Given the predicted results, of those that are predicted as positive, what proportion were True. It can be expressed mathematically as:

\begin{equation}
  Precision = \frac{True\ Positive} {(True\ Positive + False\ Positive)}
\end{equation}

Meanwhile, \textbf{Recall} measures the correctly predicted Positive class (True Positive) against both the correct and incorrect predicton on the Positive reference class (True Positive + False Negative) i.e. Given the predicted results, of those that are referenced as positive, what proportion of those were True. It can be expressed mathematically as:

\begin{equation}
  Recall = \frac{True\ Positive} {(True\ Positive + False\ Negative)}
\end{equation}

\textbf{Specificity}, aka. True-Negative-Rate measures correctly predicted Negative class (True Negative) against the correct and incorrect prediction on the Negative reference class (False Positive + True Negative) i.e. Given the predicted results, of those that are referenced as negative, what proportion of those were True. It can be expressed mathematically as:

\begin{equation}
  Specificity = \frac{True\ Negative} {(False\ Positive + True\ Negative)}
\end{equation}

Therefore, higher \textbf{Recall} suggests the model is better at identifying positives and vice-versa higher \textbf{Specificity} suggests the model is better at identifying negatives. Since this is an exercise that aim to maximise the positive prediction as a binary building segmentation classifier, emphasise will be placed on maximising \textbf{Precision} and \textbf{Recall}.

\subsection{Overall Accuracy, Dice Score, and Intersection-over-Union}\label{2ndorder}
\section{Experimentation setup}\label{ExpSetup}

Each network architecture and their associated weights will be trained on 2 data setup. The first setup consist of only the Kalobeyei, Kakuma camp where the labels include drone motion artefacts and rooftops are relatively homogeneous. The second setup consist of data from the Kalobeyei camp and also the rest of Dzaleka, Dowa camp. The second setup will introduce imperfection in labelling and complex hetereogeneous rooftops and morphologies. The two data setup will allow comparison between the models response of each class-based accuracy assessment metrics.

\subsection{Project workflow}\label{ProjWorkflow}

\newpage

\chapter{Findings}\label{findings}
\section{Analysis}\label{analysis}

\newpage

\chapter{Discussion}\label{Discuss}

\newpage

\chapter{Conclusion}\label{Conclude}

\newpage

\chapter{Bibliography}\label{Bib}

\newpage

\appendix

\chapter{Appendix}\label{Appen}

\end{document}
